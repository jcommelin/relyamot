\documentclass[a4paper,10pt]{article}

\def\fixme{\textbf}

\usepackage[]{xparse}
\usepackage[]{microtype}

\usepackage[]{amsmath,amssymb}

\usepackage[]{mathrsfs,bbold}

\usepackage[]{cleveref}

\usepackage[]{tikz-cd}

\usepackage[inline]{enumitem}

\newenvironment{lemma}{\medskip\textit{Lemma.} --- }{}
\newenvironment{proof}{\textit{Proof.} --- }{\medskip}
\newenvironment{remark}{\medskip\textit{Remark.} --- }{\medskip}

\makeatletter % {{{-
\newcommand*\map@arrow[1][]{\csname#1rightarrow\endcsname{}}

\DeclareDocumentCommand \map {o m o m o o}
{%
	\mathchoice
	{
		\IfValueTF{#5}{%
			\IfValueTF{#6}{%
				\IfValueTF{#1}{
					\arraycolsep=1.4pt
					\begin{array}{crll}
						#1\colon &
						#2 &
						\map@arrow[long\IfValueTF{#3}{#3}{}] &
						#4 \\
						& #5 & \longmapsto & #6
					\end{array}
				}{
					\arraycolsep=1.4pt
					\begin{array}{rll}
						#2 &
						\map@arrow[long\IfValueTF{#3}{#3}{}] &
						#4 \\
						#5 & \longmapsto & #6
					\end{array}
				}
			}{%
				\IfValueTF{#1}{#1\colon}{}
				#2 \map@arrow[long\IfValueTF{#3}{#3}{}] #4
			}
		}{%
			\IfValueTF{#1}{#1\colon}{}
			#2 \map@arrow[long\IfValueTF{#3}{#3}{}] #4
		}
	}
	{
		\IfValueTF{#1}{#1\colon}{}
		#2\map@arrow[\IfValueTF{#3}{#3}{}]#4%
		\IfValueTF{#5}{\IfValueTF{#6}{,#5\mapsto#6}{}}{}
	}
	{
		\IfValueTF{#1}{#1\colon}{}
		#2\map@arrow[\IfValueTF{#3}{#3}{}]#4%
		\IfValueTF{#5}{\IfValueTF{#6}{,#5\mapsto#6}{}}{}
	}
	{
		\IfValueTF{#1}{#1\colon}{}
		#2\map@arrow[\IfValueTF{#3}{#3}{}]#4%
		\IfValueTF{#5}{\IfValueTF{#6}{,#5\mapsto#6}{}}{}
	}
}
\makeatother % -}}}

\newcommand*\longhookrightarrow{\ensuremath{\lhook\joinrel\relbar\joinrel\rightarrow}}
\newcommand*\longtwoheadrightarrow{\ensuremath{\relbar\joinrel\twoheadrightarrow}}

\def\basepieces{\mathscr{V}}
\def\smpr{\mathrm{SmPr}}
\newcommand{\fgrep}[2]{#1\textrm{-}\mathrm{FGRep}_{#2}}
\def\coh{\mathrm{H}}
\def\tensor{\otimes}
\def\CH{\mathrm{CH}}
\def\pr{\mathrm{pr}}
\def\cyc{\mathscr{Z}}
\def\cycmot{A_{\mathrm{mot}}}
\def\cormot{C_{\mathrm{mot}}}
\def\Tr{\mathrm{Tr}}

\title{On relative YA-motives}
\author{Johan Commelin}

\begin{document}
\maketitle

Let $S/k$ be a smooth projective scheme over a field $k$. Let
$\map[f]{X}{S}$ be a smooth projective $S$-scheme. In some sense, one
wants to interpret $\mathrm{R}f_{*}\mathbb{Q}$ as a relative motive $h(X/S)$,
and $\mathrm{R}^{q}f_{*}\mathbb{Q}$ as the $q$-th K\"{u}nneth component
$h^{q}(X/S)$.

Now, if we drop out of the setting relative to $S$, back to the absolute
setting over $k$, we have the following decomposition, due to Deligne:
Since $f$ is smooth and projective, the Leray spectral sequence
\[
	E_{2}^{pq} = \coh^{p}(S, \mathrm{R}^{q}f_{*}\mathbb{Q})
	\quad \Longrightarrow \coh^{p+q}(X, \mathbb{Q})
\]
degenerates at the second page, and the filtration splits (non-canonically)
giving a decomposition
\[
	\coh^{n}(X, \mathbb{Q}) \cong \bigoplus_{p+q = n}
	\coh^{p}(S, \mathrm{R}^{q}f_{*}\mathbb{Q}).
\]

This decomposition is a consequence of the \emph{relative hard Lefschetz
theorem}, which is conjectured to be motivic. Yves Andr\'{e} constructed a
category of motives that accounts for this problem. In this category we give
the motivic analogue of the above decomposition. This result has as application
that the decomposition on the level of cohomology (the realisations) respects
extra structure (say Hodge structure, Galois action, \&c).

\section{Notation} \label{notation}

Let $S$ be a connected scheme. Then $\smpr_{S}$ denotes the category of smooth
projective $S$-schemes. Let $\basepieces$ be a full subcategory of $\smpr_{S}$,
such that $\basepieces$ is stable under:
\begin{enumerate*}[label=(\alph*)] % {{{-
	\item products (i.e., the fibre product over $S$);
	\item disjoint sums;
	\item taking connected components.
\end{enumerate*} % -}}}
The objects of $\basepieces$ are called \emph{base pieces}.

Let $F$ be a field of characteristic $0$. Denote with $\fgrep{F}{S}$ the
category of finite-dimensional $\mathbb{Z}$-graded $\pi(S)$-representations
over $F$. In other words, an object $A$ of $\fgrep{F}{S}$ is

\section{Relative Weil cohomology}

Let $S$ be a connected scheme. Let $F$ be a field of characteristic $0$. Recall
the notation $\smpr_{S}$, $\basepieces$, and $\fgrep{F}{S}$.

A \emph{graded commutative monoid} $A$ in $\fgrep{F}{S}$ by definition comes
with
\begin{enumerate*}[label=(\alph*)] % {{{-
	\item a $\pi(S)$-equivariant \emph{multiplication map} $A \tensor A \to
		A$ (satisfying associativity);
	\item an element $1 \in A$ that is a unit for the multiplication;
\end{enumerate*} % -}}}
such that for $x \in A^{i}$ and $y \in A^{j}$ the \emph{graded commutativity
law}
\[
	x \tensor y = (-1)^{ij} y \tensor x
\]
is satisfied.

A \emph{Weil cohomology} (on $\basepieces$ with coefficients in $F$) consists
of
\begin{enumerate}[label=(D\arabic*)] % {{{-
	\item a functor
		\[
			\map[\coh]%
			{\basepieces^{\mathrm{op}}}%
			{\fgrep{F}{S}};
		\]
	\item a $1$-dimensional graded $\pi(S)$-representation $F(1)$,
		concentrated in degree $-2$ (this gives rise to \emph{Tate
		twists}: for any $V \in \fgrep{F}{S}$, and $n \in \mathbb{Z}$
		we define $V(n)$ as $V \tensor F(1)^{\tensor n}$);
	\item (\emph{trace map}) for every $X \in \basepieces$ of (relative)
		dimension $n$ over $S$ a map
		$\map[\Tr_{X}]{\coh^{2n}(X)(n)}{F}$;
	\item (\emph{cycle map}) for every $X \in \basepieces$, and every $i
		\in \mathbb{Z}$ a map
		$\map[\gamma_{X}] {\cyc^{i}(X)}{\coh^{2i}(X)(i)^{\pi(S)}}$;
\end{enumerate} % -}}}
such that the following conditions are satisfied:
\begin{enumerate}[label=(C\arabic*)] % {{{-
	\item $\coh$ maps disjoint unions to direct sums;
	\item (\emph{cup product}) for connected $X \in \basepieces$, $\coh(X)$
		is a graded commutative monoid in $\fgrep{F}{S}$ (the
		multiplication is denoted with $\cup$);
	\item for every map $\map[f]{X}{Y}$ in $\basepieces$, with $X$ and $Y$
		connected, the induced map $f^{*} =
		\map[\coh(f)]{\coh(Y)}{\coh(X)}$ respects the unit and
		multiplication of the monoids;
	\item for $X \in \basepieces$ of (relative) dimension $n$ over $S$, and
		for $i \in \mathbb{Z}$, the $i$-th graded piece $\coh^{i}(X)$
		vanishes if $i < 0$ or $i > 2n$;
	\item (\emph{Poincar\'{e} duality}) for $X \in \basepieces$ of
		(relative dimension $n$, and for all $i \in \mathbb{Z}$ the
		composition
		\[
			\map[\Tr_{X} \circ \cup]%
			{\coh^{i}(X) \times \coh^{2n-i}(X)(n)}%
			{\coh^{2n}(X)(n) \longrightarrow F}
		\]
		induces a perfect duality between $\coh^{i}(X)$ and
		$\coh^{2n-i}(X)(n)$;
	\item (\emph{K\"{u}nneth}) for connected $X,Y \in \basepieces$ the
		canonical map
		\[
			\map{\coh(X) \tensor \coh(Y)}{\coh(X \times Y)}%
			[x \tensor y][\pr_{X}^{*} \cup \pr_{Y}^{*}]
		\]
		is an isomorphism of graded commutative monoids;
	\item for connected $X,Y \in \basepieces$ of (relative) dimension $m,n$
		over $S$, the natural diagram
		\[ \begin{tikzcd}[column sep=large] % {{{-
			\coh^{2m}(X)(m) \tensor \coh^{2n}(Y)(n)
			\drar{\Tr_{X} \tensor \Tr_{Y}}
			\dar{} & \\
			\coh^{2(m+n)}(X \times Y)(m+n)
			\rar[swap]{\Tr_{X \times Y}} &
			F
		\end{tikzcd} \] % -}}}
		is commutative;
	\item for connected $X,Y \in \basepieces$, and $i,j \in \mathbb{Z}$,
		the natural diagram
		\[ \begin{tikzcd}[column sep=large] % {{{-
			\cyc^{i}(X) \tensor \cyc^{j}(Y) \dar
			\rar{\gamma_{X} \tensor \gamma_{Y}} &
			\coh^{2i}(X)(i) \tensor \coh^{2j}(Y)(j) \dar \\
			\cyc^{i+j}(X \times_{S} Y) \rar{\gamma_{X \times Y}} &
			\coh^{2(i+j)}(X \times Y)(i+j)
		\end{tikzcd} \] % -}}}
		is commutative;
	\item for a map $\map[f]{X}{Y}$ in $\basepieces$, a prime cycle
		$x \in \cyc^{i}(X)$, the integer $d$ satisfying $f_{*}x =
		d\overline{f[x]}$ (i.e., the degree of the map
		$\map{x}{\overline{f[x]}}$), and a class $\alpha \in
		\coh^{2(n-i)}(Y)(n-i)$, the equality
		\[
			\Tr_{X}(\gamma{X}(x) \cup f^{*}\alpha) =
			d\Tr_{Y}(\gamma_{Y}(f_{*}x) \cup \alpha)
		\]
		holds;
	\item for a map $\map[f]{X}{Y}$ in $\basepieces$, the equality
		\[
			\gamma_{X} \circ f^{*} = f^{*} \circ \gamma_{Y}
		\]
		holds (i.e., the cycle map commutes with pullbacks);
	\item for the base scheme $S$, the image of $\Tr_{S} \circ \gamma_{S}$
		is $\mathbb{Z} \subset F$.
\end{enumerate} % -}}}

\begin{remark} % {{{-
	We deliberately choose $\fgrep{F}{S}$ as target of $\coh$, instead of
	the category of graded commutative monoids in $\fgrep{F}{S}$. This way,
	we later have a very natural extension of $\coh$ to motives.
\end{remark} % -}}}

A relative Weil cohomology is said to satisfy the \emph{weak Lefschetz theorem}
if for each $X \in \basepieces$ of (relative) dimension $n$ over $S$, and each
smooth hyperplane section $Y \in \basepieces$ of $X$ the maps on cohomology
induced by $\map{Y}[hook]{X}$ satisfy
\[
	\map{\coh^{i}(X)}{\coh^{i}(Y)} \text{ is }
	\begin{cases} % {{{-
		\text{an isomorphism} & \text{if } 0 \le i < n - 1 \\
		\text{injective} & \text{if } i = n - 1.
	\end{cases} % -}}}
\]

A relative Weil cohomology is said to satisfy the \emph{hard Lefschetz theorem}
if for each $X \in \basepieces$ of (relative) dimension $n$ over $S$, and each
smooth hyperplane section $Y \in \basepieces$ of $X$ (in particular $Y \in
\cyc^{1}(X)$) the map of degree $2$
\[
	\map[L]{\coh^{\bullet}(X)}{\coh^{\bullet+2}(X)}%
	[\alpha][\alpha \cup \gamma_{X}(Y)]
\]
induces isomorphisms
\[
	\map[L^{i}]{\coh^{n-i}(X)}{\coh^{n+i}(X)}.
\]

\begin{lemma} % {{{-
	The classical Weil cohomologies have relative versions.
	\begin{enumerate} % {{{-
		\item Let $S$ be a smooth projective $\mathbb{C}$-scheme. Then
			\[
				\map[\coh]%
				{\basepieces^{\mathrm{op}}}%
				{\fgrep{\mathbb{Q}}{S}}%
				[\scriptstyle(\map[f]{X}{S})]%
				[\mathrm{R}f_{*}\mathbb{Q}]
			\]
			is a relative Weil cohomology satisfying the weak and
			hard Lefschetz theorems.
		\item Let $k$ be a field (say global, local, or finite), and
			let $S$ be a smooth projective $k$-scheme. Let $\ell$
			be a prime different from the characteristic of $k$.
			Then
			\[
				\map[\coh]%
				{\basepieces^{\mathrm{op}}}%
				{\fgrep{\mathbb{Q_{\ell}}}{S}}%
				[\scriptstyle(\map[f]{X}{S})]%
				[\mathrm{R}f_{*}\mathbb{Q}_{\ell}]
			\]
			is a relative Weil cohomology satisfying the weak and
			hard Lefschetz theorems.
		\item Let $k$ be a field, and let $S$ be a smooth projective
			$k$-scheme. (Version for algebraic de Rham.)
		\item (Version for crystalline???)
	\end{enumerate} % -}}}
	\begin{proof} % {{{-
		Hopefully this is not too hard.
	\end{proof} % -}}}
\end{lemma} % -}}}

\section{Relative motives}

%Define the category of relative YA-motives: $\mathcal{M}_{S}$ the category of
%YA-motives over $S$.
Let $S$ be a connected scheme, and $\basepieces$ a category of \emph{base
pieces} (as in \cref{notation}). Fix a reference cohomology $\coh$ on
$\basepieces$, with coefficients in a field $F$ of characteristic $0$,
satisfying the weak and hard Lefschetz theorems.

Denote with $\pr^{XY}_{X}$ the projection $\map{X \times Y}{X}$. Let $\star$
denote either the Lefschetz or the Hodge star operator. Let $E$ be a subfield
of $F$.

A \emph{motivated cycle over $X$ with coefficients in $E$} is an element of
$\coh(X)$ of the form $\pr^{XY}_{X}(\alpha \cup \star \beta)$, where:
\begin{itemize} % {{{-
	\item $\alpha$ and $\beta$ are algebraic cycles with coefficients in
		$E$, over $X \times Y$, for some arbitrary $Y \in \basepieces$,
	\item $\star = \star_{XY}$ is relative to $L_{X} \tensor L_{Y}$, where
		$L_{X}$ (resp.\ $L_{Y}$) is a Lefschetz operator on $X$ (resp.\
		$Y$).
\end{itemize} % -}}}

Denote with $\cycmot(X)_{E}$ (or $\cycmot(X)_{E}^{\coh}$ if the reference
cohomology is ambiguous) the set of motivated cycles over $X$ with coefficients
in $E$.

\begin{lemma} % {{{-
	\begin{enumerate*}[label=(\alph*)] % {{{-
		\item Its a sub-$E$-algebra;
		\item Contains $A(X)$ and $\star A(X)$;
		\item Behaves well under pullback and pushforward.
	\end{enumerate*} % -}}}

	\begin{proof} % {{{-
		\fixme{Ought to be the same as in Andr\'{e}.}
	\end{proof} % -}}}
\end{lemma} % -}}}

Let $X = \coprod_{i} X_{i}$ and $Y = \coprod_{i} Y_{j}$ be two objects of
$\basepieces$, with their decompositions in connected components. Let $r =
(r_{ij})_{ij}$ be a locally constant $\mathbb{Z}$-valued function on $X
\times_{S} Y$ (i.e., an integer $r_{ij}$ on each $X_{i} \times_{S} Y_{j}$). A
\emph{motivated correspondence from $X$ to $Y$ of degree $r$ with coefficients
in $E$} is an element of $\oplus_{ij} \cycmot^{\dim(X_{i}) + r_{ij}}(X_{i}
\times_{S} Y_{j})_{E}$. Denote with $\cormot(X,Y)_{E}$ the graded space of
motivated correspondences from $X$ to $Y$ with coefficients in $E$.

\begin{lemma} % {{{-
	When composing motivated correspondences, their degrees add up. In
	particular $\cormot(X,X)_{E}$ is a graded $E$-algebra.

	\begin{proof} % {{{-
		\fixme{Ought to be the same as in Andr\'{e}.}
	\end{proof} % -}}}
\end{lemma} % -}}}

For morphisms $f$, we get $f_{*}$ (resp. $f^{*}$), by composing with the graph
of $f$ (resp. its transpose). They are related by $f_{*}(\alpha \cup f^{*}
\beta) = f_{*}(\alpha) \cup \beta$.

\begin{lemma} % {{{-
	The algebra $\cormot(X,X)_{E}$ contains the Lefschetz and Hodge star
	operators, and the K\"{u}nneth projectors.

	\begin{proof} % {{{-
		\fixme{Ought to be the same as in Andr\'{e}.}
	\end{proof} % -}}}
\end{lemma} % -}}}

\fixme{We need a notion of compatibility for relative Weil cohomologies over
	$S'$, and over $S$, whenever there is some smooth projective
$\map{S'}{S}$.}

\fixme{We want some pushforward and pullback of motives along a smooth
projective $\map{S'}{S}$. Also, an adjunction?}

% Crucial lemma
\begin{lemma} % {{{-
	Let $S' \in \basepieces$ be some smooth projective $S$-scheme, and
	assume that $\basepieces$ is closed under fibre porducts over $S'$. Let
	$\basepieces'$ be the slice category of $S'$-schemes in $\basepieces$.
	The category $\basepieces'$ is a category of base pieces in
	$\smpr_{S'}$.

	Let $\mathcal{M}_{S}(\basepieces)'$ be the coslice category of
	$h(S')$-motives in $\mathcal{M}_{S}(\basepieces)$. (I.e., objects:
	$h(S') \to M$ in $\mathcal{M}_{S}(\basepieces)$; morphisms: commutative
	triangles in $\mathcal{M}_{S}(\basepieces)$).

	The functor
	\[
		\map{\mathcal{M}_{S'}(\basepieces')}%
		{\mathcal{M}_{S}(\basepieces)}%
		[(X/S', p, m)][(X/S, p, m)]
	\]
	is well-defined and induces and equivalence of
	$\mathcal{M}_{S'}(\basepieces')$ with $\mathcal{M}_{S}(\basepieces)'$.
\end{lemma} % -}}}


\section{Decomposition theorem}

\section{Wishful thinking}

What we really want is a motivic decomposition theorem. So, for some smooth
projective maps $\map[g]{X}{S}$ and $\map[f]{S}{\mathrm{Spec}(k)}$, a statement
that looks like
\[
	\mathrm{E}_{2}^{p,q} = h^{p}(\mathrm{R}^{q}g_{*} \mathbb{1}_{X})
	\quad \Longrightarrow h^{p+q}(X)
\]
and
\[
	h^{n}(X) \cong \bigoplus_{p+q = n} h^{p}(\mathrm{R}^{q}g_{*}
	\mathbb{1}_{X}).
\]
(Here $\mathbb{1}_{X}$ denotes the unit motive over $X$.)

What we need for this, is:
\begin{enumerate} % {{{-
	\item a theory of relative motives;
	\item the K\"{u}nneth projectors should be motivic;
	\item a pushforward of motives along smooth projective maps.
\end{enumerate} % -}}}
A good candidate would be a generalisation of Andr\'{e} motives to the relative
setting. The pushforward along $\map{S'}{S}$ would for example be defined as
\[
	(X/S', p, n) \mapsto (X/S, \Delta^{S'}_{S,*}(p), n)
\]
where $\Delta^{S'}_{S}$ is the canonical map $\map{X \times_{S'} X}{X
\times_{S} X}$.

However, to generalise Andr\'{e} motives to the relative setting, we need to
generalise Weil cohomology theories to the relative setting. Moreover, to make
sense of the pushforward, we need to be able to relate motives over $S'$ with
motives over $S$; and therefore also relative Weil cohomology over $S'$ with
relative Weil cohomology over $S$. This suggests that we should assign in a
functorial manner to each $S$ a relative Weil cohomology theory over $S$.

\subsection{A fibred category approach to the problem}

Let $S$ be a scheme. Observe that $\map{(\smpr_{S})^{\Delta^{1}}}{\smpr_{S}}$
is a fibred category. For each smooth projective scheme $S'$ over $S$, we have
the fibre $\smpr_{S'}$. This fibred category is the source of the Weil
cohomology functor.

We now need to think how to encode the target. The approach above suggests that
we assign to each $S'$ a (pro-algebraic) group $\pi(S)$ (e.g., the \'{e}tale
fundamental group, or the Tannaka group associated with polarisable variations
of Hodge structures on $S$).

The decomposition theorem on the level of cohomology definitely needs derived
structures, so it makes sense that every $X$ in the fibre above $S'$ is mapped
to an object in the derived category of finite-dimensional
$\pi(S')$-representations over some field $F$.

Hence the target should be the fibred category obtained from the Grothendieck
construction
\[
	\map{\mathrm{Grp}}{\mathrm{Cat}}%
	[G][\mathcal{D}(\mathrm{FRep}_{F}(G)^{\mathrm{op}}.]
\]
More precisely, we map $\map{H}{G}$ to $\mathcal{D}(\mathrm{Ind}^{G}_{H})$, and
it is easy to that this is pseudo-functorial. In this way, we obtain a fibred
category
\[
	\map{\mathcal{D}(\mathrm{FRep}_{F}(\_))^{\mathrm{op}}}{\mathrm{Grp}}.
\]
In particular the set $\mathrm{Hom}(V_{H},W_{G})$, with $V_{H}, W_{G}$ objects
in the fibres above $H, G$ respectively, consists of pairs $(f, \phi)$, where
$f$ is a map $\map[f]{H}{G}$, and $\phi$ a morphism
$\map[\phi]{W_{G}}{\mathrm{Ind}^{G}_{H}(V_{H})}$ in the category
$\mathcal{D}(\mathrm{FRep}_{F}(G))$, i.e. the opposite of the fibre above $G$.

A Weil cohomology theory will then be a morphism of fibred categories $(\pi,
\coh)$,
\[
	\begin{tikzcd} % {{{-
		(\smpr_{S})^{\Delta^{1}} \rar{\coh} \dar &
		\mathcal{D}(\mathrm{FRep}_{F}(\_))^{\mathrm{op}} \dar \\
		\smpr_{S} \rar{\pi} & \mathrm{Grp}
	\end{tikzcd} % -}}}
\]
subject to an awful load of conditions that I still need to figure out.

Hopefully, the classical theories (Betti, de Rham, $\ell$-adic \'{e}tale,
crystalline) fit in this picture. If not, the approach is pretty much doomed.
If we succeed, generalising Andr\'{e} motives should not be too hard.

We then want the (well-formed) version of the last lemma of the previous
section:

\begin{lemma} % {{{-
	Let $\map{X}{Y}$ be a map in $\smpr_{S}$. Giving a motive over $X$ is
	the same as giving an $h(X)$-motive over $Y$.
\end{lemma} % -}}}

\section{Relative Chow motives}

We refer to [MNP] for the definition of relative Chow motives. Let $S$ be a
scheme, and let $\mathcal{M}_{S}$ denote the category of relative Chow motives
over $S$. Let $\map[h_{S}]{\smpr_{S}^{\mathrm{op}}}{\mathcal{M}_{S}}$ be the
functor mapping $X$ to the motive $(X, \mathrm{id}, 0)$.
	
Let $\map[f]{S}{S'}$ be a smooth projective morphism. As can be read in [MNP,
Cor 8.1.7], there is a functor
\[
	\map[f_{*}]{\mathcal{M}_{S}}{\mathcal{M}_{S'}}%
	[(X,p,m)][(X,\Delta^{S}_{S',*}(p),m)]
\]
where $\map[\Delta^{S}_{S'}]{X \times_{S} X}{X \times_{S'} X}$ is the canonical
map induced by projections to $X$.

Observe that the unit object $h_{S}(S) \in \mathcal{M}_{S}$ is mapped to
$h_{S'}(S)$ in $\mathcal{M}_{S'}$.

Further, the functor is additive: this boils down to the push-forward on Chow
groups along $\map{X \times_{S} Y}{X \times_{S'} Y}$ being a group
homomorphism.

Define a cohomology functor
\[
	\map[\coh_{S}]{\mathcal{M}_{S}}%
	{\text{$\mathbb{Q}_{\ell}$-sheaves on $S$}}%
	[\left((\map[g]{X}{S}),p,m\right)]%
	[\mathrm{Im}(p_{*}|\mathrm{R}g_{*}\mathbb{Q}_{\ell})(m)]
\]
We should also indicate what $\coh_{S}$ does with morphisms. Let
$\map[g]{X}{S}$ and $\map[h]{Y}{S}$ be two object in $\smpr_{S}$. First observe
that if $\map[f]{X}{Y}$ is a morphism in $\smpr_{S}$, then we have induced
morphisms
$\map[f_{*}]{\mathrm{R}g_{*}\mathbb{Q}_{\ell}}{\mathrm{R}h_{*}\mathbb{Q}_{\ell}}$
and
$\map[f^{*}]{\mathrm{R}h_{*}\mathbb{Q}_{\ell}}{\mathrm{R}g_{*}\mathbb{Q}_{\ell}}$.
One can see this as follows: take $s \in S$. Then
$(\mathrm{R}g_{*}\mathbb{Q}_{\ell})_{s}$ is the cohomology of $X_{s}$ (and
similarly for $Y$). The stalk $(\mathrm{R}g_{*}\mathbb{Q}_{\ell})_{s}$
determines lisse $\ell$-adic sheaf $\mathrm{R}g_{*}\mathbb{Q}_{\ell}$, and the
$f_{*}$ and $f^{*}$ morphisms are induced by those between the cohomology of
$X_{s}$ and $Y_{s}$.

Let $\map[p]{X}{Y}$ be a correspondence over $S$.

We now want to show that the following diagram commutes.
\[
	\begin{tikzcd} % {{{-
		\mathcal{M}_{S} \rar{f_{*}} \dar{\coh_{S}}
		& \mathcal{M}_{S'} \dar{\coh_{S'}} \\
		\text{$\mathbb{Q}_{\ell}$-sheaves on $S$} \rar{\mathrm{R}f_{*}}
		& \text{$\mathbb{Q}_{\ell}$-sheaves on $S'$}
	\end{tikzcd} % -}}}
\]
It is clear that this commutes on objects $h(X) = (X, \Delta, 0)$.

\end{document}
