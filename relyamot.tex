\documentclass[a4paper,10pt]{article}

\def\fixme{\textbf}

\usepackage[]{amsmath,amssymb}
\usepackage[]{mathrsfs}

\usepackage[inline]{enumitem}

\newenvironment{lemma}{\textit{Lemma.} --- }{}
\newenvironment{remark}{\textit{Remark.} --- }{}

\makeatletter % {{{-
\newcommand*\map@arrow[1][]{\csname#1rightarrow\endcsname{}}

\newcommand*\function@textstyle[4][]{#2\colon#3\map@arrow[#1]#4}
\newcommand*\function[4][]{%
	\mathchoice
	{\function@textstyle[long#1]{#2}{#3}{#4}}
	{\function@textstyle[#1]{#2}{#3}{#4}}
	{\function@textstyle[#1]{#2}{#3}{#4}}
	{\function@textstyle[#1]{#2}{#3}{#4}}
}
\newcommand*\map@textstyle[6][]{#2\colon#3\rightarrow#4:#5\mapsto#6}
\newcommand*\map[6][]{%
	\mathchoice
	{#2\colon\
		\begin{array}{ccc}
			#3&~\map@arrow[long#1]~&#4 \\
			#5&~\longmapsto~&#6
		\end{array}
	}
	{\map@textstyle[#1]{#2}{#3}{#4}{#5}{#6}}
	{\map@textstyle[#1]{#2}{#3}{#4}{#5}{#6}}
	{\map@textstyle[#1]{#2}{#3}{#4}{#5}{#6}}
}
\makeatother % -}}}

\newcommand*\longhookrightarrow{\ensuremath{\lhook\joinrel\relbar\joinrel\rightarrow}}
\newcommand*\longtwoheadrightarrow{\ensuremath{\relbar\joinrel\twoheadrightarrow}}

\def\basepieces{\mathscr{V}}
\def\smpr{\mathrm{SmPr}}
\newcommand{\fgrep}[2]{#1\mathrm{-FGRep}_{#2}}
\def\coh{\mathrm{H}}
\def\tensor{\otimes}
\def\CH{\mathrm{CH}}
\def\pr{\mathrm{pr}}

\title{On relative YA-motives}
\author{Johan Commelin}

\begin{document}
\maketitle

Let $S/k$ be a smooth projective scheme over a field $k$. Let
$\function{f}{X}{S}$ be a smooth projective $S$-scheme. In some sense, one
wants to interpret $\mathrm{R}f_{*}\mathbb{Q}$ as a relative motive $h(X/S)$,
and $\mathrm{R}^{q}f_{*}\mathbb{Q}$ as the $q$-th K\"{u}nneth component
$h^{q}(X/S)$.

Now, if we drop out of the setting relative to $S$, back to the absolute
setting over $k$, we have the following decomposition, due to Deligne:
Since $f$ is smooth and projective, the Leray spectral sequence
\[
	E_{2}^{pq} = \coh^{p}(S, \mathrm{R}^{q}f_{*}\mathbb{Q})
	\quad \Rightarrow \coh^{p+q}(X, \mathbb{Q})
\]
degenerates at the second page, and the filtration splits (non-canonically)
giving a decomposition
\[
	\coh^{n}(X, \mathbb{Q}) \cong \bigoplus_{p+q = n}
	\coh^{p}(S, \mathrm{R}^{q}f_{*}\mathbb{Q}).
\]

This decomposition is a consequence of the \emph{relative hard Lefschetz
theorem}, which is conjectured to be motivic. Yves Andr\'{e} constructed a
category of motives that accounts for this problem. In this category we give
the motivic analogue of the above decomposition. This result has as application
that the decomposition on the level of cohomology (the realisations) respects
extra structure (say Hodge structure, Galois action, \&c).

\section{Notation}

Let $S$ be a connected scheme. Then $\smpr_{S}$ denotes the category of smooth
projective $S$-schemes with geometrically integral fibres. Let $\basepieces$ be
a full subcategory of $\smpr_{S}$, such that $\basepieces$ is stable under:
\begin{enumerate*}[label=(\alph*)] % {{{-
	\item products (i.e., the fibre product over $S$);
	\item disjoint sums;
	\item taking connected components.
\end{enumerate*} % -}}}
The objects of $\basepieces$ are called \emph{base pieces}.

Let $F$ be a field of characteristic $0$. Denote with $\fgrep{F}{S}$ the
category of finite-dimensional $\mathbb{Z}$-graded $\pi(S)$-representations
over $F$. In other words, an object $A$ of $\fgrep{F}{S}$ is

\section{Relative algebraic cycles}

\section{Relative Weil cohomology}

Let $S$ be a connected scheme. Let $F$ be a field of characteristic $0$. Recall
the notation $\smpr_{S}$, $\basepieces$, and $\fgrep{F}{S}$.

A \emph{graded commutative monoid} $A$ in $\fgrep{F}{S}$ by definition comes
with
\begin{enumerate*}[label=(\alph*)] % {{{-
	\item a $\pi(S)$-equivariant \emph{multiplication map} $A \tensor A \to
		A$ (satisfying associativity);
	\item an element $1 \in A$ that is a unit for the multiplication;
\end{enumerate*} % -}}}
such that for $x \in A^{i}$ and $y \in A^{j}$ the \emph{graded commutativity
law}
\[
	x \tensor y = (-1)^{ij} y \tensor x
\]
is satisfied.

A \emph{Weil cohomology} (on $\basepieces$ with coefficients in $F$) consists
of
\begin{enumerate}[label=(D\arabic*)] % {{{-
	\item a functor
		\[
			\function{\coh}%
			{\basepieces^{\mathrm{op}}}%
			{\fgrep{F}{S}};
		\]
	\item a $1$-dimensional graded $\pi(S)$-representation $F(1)$,
		concentrated in degree $-2$ (this gives rise to \emph{Tate
		twists}: for any $V \in \fgrep{F}{S}$, and $n \in \mathbb{Z}$
		we define $V(n)$ as $V \tensor F(1)^{\tensor n}$);
	\item a \emph{cycle map}
		$\function{\gamma_{X}}%
		{\CH^{i}(X/S)}%
		{\coh^{2i}(X)^{\pi(S)}(i)}$;
\end{enumerate} % -}}}
such that for $X \in \basepieces$ of (relative) dimension $n$ over $S$:
\begin{enumerate}[label=(C\arabic*)] % {{{-
	\item (\emph{Cup product}) $\coh(X)$ is a graded commutative monoid in
		$\fgrep{F}{S}$ (the multiplication is denoted with $\cup$);
	\item for every map $\function{f}{X}{Y}$ in $\basepieces$, the induced
		map $f^{*} = \function{\coh(f)}{\coh(Y)}{\coh(X)}$ is respects
		the unit and multiplication of the monoids;
	\item for $i < 0$ and for $i > 2n$ the $i$-th graded piece
		$\coh^{i}(X)$ is $0$;
	\item $\coh^{2n}(X)$ is a representation of dimension $1$ (isomorphic
		to $F(n)$ ???);
	\item (\emph{Poincar\'{e} duality}) for all $i$ there is a
		nondegenerate pairing $\coh^{i}(X) \times \coh^{2n-i}(X) \to
		\coh^{2n}(X)$;
	\item (\emph{K\"{u}nneth}) for every $Y \in \basepieces$
		the canonical map
		\[
			\map{}{\coh(X) \tensor \coh(Y)}{\coh(X \times Y)}%
			{x \tensor y}{\pr_{X}^{*} \cup \pr_{Y}^{*}}
		\]
		is an isomorphism;
	\item foo
\end{enumerate} % -}}}

\begin{remark} % {{{-
	We deliberately choose $\fgrep{F}{S}$ as target of $\coh$, instead of
	the category of graded commutative monoids in $\fgrep{F}{S}$. This way,
	we later have a very natural extension of $\coh$ to motives.
\end{remark} % -}}}

\section{Relative motives}

%Define the category of relative YA-motives: $\mathcal{M}_{S}$ the category of
%YA-motives over $S$.
Fix a reference
cohomology $\coh$ on $\basepieces$, with coefficients in a field $F$ of
characteristic $0$, satisfying the hard Lefschetz theorem (relative to $S$).

\begin{lemma} % {{{-
	Let $X/S$ be a smooth projective scheme over a scheme $S$. Write
	$\mathcal{M}_{X}$ for the category of YA-motives over $X$.

	Let $h(X)$ be the motive of $X$ in $\mathcal{M}_{S}$. Let $\mathcal{C}$
	be the under category $h(X) \downarrow \mathcal{M}_{S}$ (objects: $h(X)
	\to M$ in $\mathcal{M}_{S}$; morphisms: commutative triangles in
	$\mathcal{M}_{S}$).

	The category $\mathcal{M}_{X}$ is equivalent to $\mathcal{C}$.
	
	(Which way does the easy functor go?)
\end{lemma} % -}}}


\section{Decomposition theorem}

\end{document}
