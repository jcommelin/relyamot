\documentclass[a4paper,10pt]{article}

\usepackage[]{amsmath,amssymb}

\newenvironment{lemma}{\textit{Lemma.} --- }{}

\title{On relative YA-motives}
\author{Johan Commelin}

\begin{document} % {{{-
\maketitle

Let $S/k$ be a smooth projective scheme over a field $k$. Let $f \colon X \to
S$ be a smooth projective $S$-scheme. In some sense, one wants to interpret
$\textrm{R}f_{*}\mathbb{Q}$ as a relative motive $h(X/S)$, and
$\textrm{R}^{q}f_{*}\mathbb{Q}$ as the $q$-th K\"{u}nneth component
$h^{q}(X/S)$.

Now, if we drop out of the setting relative to $S$, back to the absolute
setting over $k$, we have the following decomposition, due to Deligne:
Since $f$ is smooth and projective, the Leray spectral sequence
\[
	E_{2}^{pq} = \textrm{H}^{p}(S, \textrm{R}^{q}f_{*}\mathbb{Q})
	\quad \Rightarrow \textrm{H}^{p+q}(X, \mathbb{Q})
\]
degenerates at the second page, and the filtration splits (non-canonically)
giving a decomposition
\[
	\textrm{H}^{n}(X, \mathbb{Q}) \cong \bigoplus_{p+q = n}
	\textrm{H}^{p}(S, \textrm{R}^{q}f_{*}\mathbb{Q}).
\]

This decomposition is a consequence of the \emph{relative hard Lefschetz
theorem}, which is conjectured to be motivic. Yves Andr\'{e} constructed a
category of motives that accounts for this problem. In this category we give
the motivic analogue of the above decomposition. This result has as application
that the decomposition on the level of cohomology (the realisations) respects
extra structure (say Hodge structure, Galois action, \&c).

\section{Relative motives}

Let $S/k$ be a smooth projective scheme over a field $k$. Define the category
of relative YA-motives: $\mathcal{M}_{S}$ the category of YA-motives over $S$.

\begin{lemma} % {{{-
	Let $S/k$ be a smooth projective scheme over a field $k$. Write
	$\mathcal{M}_{S}$ for the category of YA-motives over $S$.

	Let $h(S)$ be the motive of $S$ in $\mathcal{M}_{k}$. Let $C$ be the
	under category $h(S) \downarrow \mathcal{M}_{k}$ (objects: $h(S) \to M$
	in $\mathcal{M}_{k}$; morphisms: commutative triangles in
	$\mathcal{M}_{k}$).

	The category $\mathcal{M}_{S}$ is equivalent to $C$.
	
	(Which way does the easy functor go?)
\end{lemma} % -}}}


\section{Decomposition theorem}

\end{document} % -}}}
